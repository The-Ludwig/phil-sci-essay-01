\usepackage{fontspec}
\usepackage{amsmath}
\usepackage{amssymb}
\usepackage{mathtools}

%Physics styles
\usepackage[
  math-style=ISO,
  bold-style=ISO,
  sans-style=italic,
  nabla=upright,
  partial=upright
]{unicode-math}

% numbers and units
\usepackage[
  %   locale=DE,
  separate-uncertainty=true,  % use \pm
  per-mode=symbol-or-fraction,
  binary-units=true,
  %per-mode=reciprocal,
  %output-decimal-marker=.,
]{siunitx}

% Fix missing micro sign with TL2017
\sisetup{
  math-micro=\text{μ},
  text-micro=μ,
  range-phrase = -,
  list-separator       = {, },
  list-final-separator = { und },
  range-units = single
}

% Positioning
\usepackage{float}
% Floats inside of section 
\usepackage[section, below]{placeins}
% same for Subsections 
\makeatletter
\AtBeginDocument{%
  \expandafter\renewcommand\expandafter\subsection\expandafter{%
    \expandafter\@fb@secFB\subsection
  }%
}
\makeatother
\floatplacement{figure}{htbp}
\floatplacement{table}{htbp}

% making caption look nice
\usepackage[
  labelfont=bf,
  font=small,
  width=0.9\textwidth,
]{caption}
\usepackage{subcaption}

% pictures
\usepackage{graphicx}

% Use _ in paths
\usepackage{grffile}

% tables
\usepackage{booktabs}

% optimizing look
\usepackage{microtype}
\usepackage[shortcuts]{extdash}

\usepackage{tikz}
\usepackage{tikz-feynman}

\setmainfont{Libertinus Serif}
\setsansfont{Libertinus Sans}
\setmonofont{Libertinus Mono}
\setmathfont{Libertinus Math}
\DeclareMathAlphabet{\mathcal}{OMS}{cmsy}{m}{n}
\recalctypearea

\usepackage[
  backend=biber,   % use modern biber backend
  autolang=hyphen, % load hyphenation rules for if language of bibentry is not
  % german, has to be loaded with \setotherlanguages
  % in the references.bib use langid={en} for english sources
  urldate=iso,%
  date=iso,%
  style=phys,%
  articletitle=true,biblabel=brackets,%
  chaptertitle=false,pageranges=false,%
  maxnames=2%
]{biblatex}

\usepackage{todonotes}
\presetkeys{todonotes}{inline}{}
%\setuptodonotes{color=lhcb3,size=\tiny}
\usepackage{xfrac}

\usetikzlibrary{patterns}
\usetikzlibrary{angles}
\usepackage{tabularx}

% quotation marks
\usepackage{csquotes}
\usepackage{slashed}
\usepackage{expl3}
\usepackage{xparse}
\usepackage{xcolor}
\usepackage{ifthen}
\usepackage{mleftright}
% my makros
% useful makros
\ExplSyntaxOn

\DeclareSIUnit\century{century}
\DeclareSIUnit\year{yr}

\makeatletter % allows me to use @
\NewDocumentCommand \showfont {} 
{encoding: \f@encoding{},
  family: \f@family{},
  series: \f@series{},
  shape: \f@shape{},
  size: \f@size{}
}
\NewDocumentCommand \symbfiftextbf {m}
{
    \ifthenelse{\equal{\f@series}{bx}\or\equal{\f@series}{b}}{\symbf{#1}}{#1}
}

\makeatother

\AtBeginDocument{
\let\ltext=\l
\RenewDocumentCommand \l {}
{
    \TextOrMath{ \ltext }{ \mleft }
}
\let\raccent=\r
\RenewDocumentCommand \r {}
{
    \TextOrMath{ \raccent }{ \mright }
}
}
\NewDocumentCommand \dif {}
{
    \mathinner{\symup{d}\mathchoice{\!}{\!}{}{} }
}

\ExplSyntaxOff
\NewDocumentCommand \tindex {mm}
{
    {#1_{\symup{#2}}}
}
\ExplSyntaxOn

\setlength{\delimitershortfall}{-1sp}
\DeclarePairedDelimiter{\abs}{\lvert}{\rvert}
\DeclarePairedDelimiter{\norm}{\lVert}{\rVert}
\DeclarePairedDelimiter\bra{\langle}{\rvert}
\DeclarePairedDelimiter\ket{\lvert}{\rangle}
\DeclarePairedDelimiterX\braket[2]{\langle}{\rangle}{#1 \delimsize\vert #2}

\NewDocumentCommand\xDeclarePairedDelimeter{mmm}
{%
\NewDocumentCommand#1{som}{%
\IfNoValueTF{##2}
    {\IfBooleanTF{##1}{#2##3#3}{\mleft#2##3\mright#3}}
{\mathopen{##2#2}##3\mathclose{##2#3}}%
}%
}
\xDeclarePairedDelimeter{\set}{\lbrace}{\rbrace}

\let\mysubsection=\subsection
\RenewDocumentCommand\subsection{m}
{
    \FloatBarrier
    \mysubsection{#1}
}

\NewDocumentCommand{\anti} {m}
{
    \overline{#1}
}

\AtBeginDocument{
    \RenewDocumentCommand \Re {} {\operatorname{Re}}
    \RenewDocumentCommand \Im {} {\operatorname{Im}}
}

\NewDocumentCommand{\mat}{m}{
    \symbf{#1}
}

\ExplSyntaxOff


% to use the LHCb colors, the same as in the plots
\definecolor{lhcb1}{HTML}{1f77b4}
\definecolor{lhcb2}{HTML}{ff7f0e}
\definecolor{lhcb3}{HTML}{2ca02c}
\definecolor{lhcb4}{HTML}{d62728}
\definecolor{lhcb5}{HTML}{9467bd}
\definecolor{lhcb6}{HTML}{8c564b}
\definecolor{lhcb7}{HTML}{e377c2}
\definecolor{lhcb8}{HTML}{7f7f7f}
\definecolor{lhcb9}{HTML}{bcbd22}
\definecolor{lhcb10}{HTML}{17becf}

\definecolor{mDarkBrown}{HTML}{604c38}
\definecolor{mDarkTeal}{HTML}{23373b}
\definecolor{mLightBrown}{HTML}{EB811B}
\definecolor{mLightGreen}{HTML}{14B03D}
\definecolor{vertexDarkGrey}{HTML}{23373b}
\definecolor{vertexLightGrey}{rgb}{0.833333333,0.8117647064,0.790196078}
\colorlet{vertexDarkRed}{red!70!black}
\colorlet{vertexLightRed}{vertexDarkRed!60!white}
\colorlet{vertexDarkGreen}{green!70!black}
\colorlet{vertexLightGreen}{vertexDarkGreen!50!white}
\colorlet{vertexDarkBlue}{blue!70!black}
\colorlet{vertexLightBlue}{vertexDarkBlue!50!white}


% Hyperlinks
\usepackage[
  allbordercolors = vertexDarkGreen,
  unicode,        % Unicode in PDF-Attributen erlauben
  pdfusetitle,    % Titel, Autoren und Datum als PDF-Attribute
  pdfcreator={},  % ┐ PDF-Attribute säubern
  pdfproducer={}, % ┘
]{hyperref}
% erweiterte Bookmarks im PDF
\usepackage{bookmark}
