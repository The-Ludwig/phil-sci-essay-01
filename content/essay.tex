\section*{Introduction}

The role of so-called \emph{Laws of Nature} is of great interest
to the philosophy of science.
Intuitively, it seems as if every natural science has laws:
Prominent examples include Newton's \emph{law} of gravitation in physics,
the laws of covalent bonding in chemistry or Mendel's law in biology.
Aside from the fact that these powerful scientific concepts are called \enquote{laws},
many scientists also work under the assumption of laws, or
even actively try to discover new ones.
Thus, it is an important role of the philosophy of science to define and understand
laws of nature and their implication, truth content and usefulness.

Logical positivism defines a law of nature as a universal, necessary statement, which is not analytically true.
This definition is widely agreed upon \cite[57]{philsciencebook}.
The features of universality and being non-analytic are rather intuitive:
After all, the statement \enquote{Some atoms are radioactive.} (which is not universal) doesn't tell us if
Radon is radioactive and \enquote{Every atom contains one or more protons.} (which is analytic)
is just a description of the meaning of \enquote{atom}.
(Nomic) Necessity means statements are not accidentally true,
by the concrete circumstances of reality
(like \enquote{The most common atom in the universe is hydrogen.}, which is not-necessary).
To distinguish between necessary and not-necessary, we can also require that
the laws of nature supports \emph{counterfactuals}, which are hypothetical if-scenarios
to test the law
In our not-necessary atom-law example, the counterfactual
\enquote{If all hydrogen atoms were fused to helium, hydrogen would be the most common element in the universe.}
is false, since it is just a coincident law.
Laws of nature are required in \emph{covering law} (c.l.) models, like the \enquote{deductive-nomological (DN) model}
or the \enquote{inductive-statistical (IS) model} \cite[Ch. 3]{philsciencebook}.
This is a class of models which define a scientific explanation as
always using some law of nature, which is relevant to (covers) the to-be-explained phenomenon
(the explanandum).

Nancy Cartwright argues in her 1983 essay \enquote{The Truth Doesn't Explain Much}
that science does not use real laws of nature at all, rather that the explanation
is made using \emph{ceteris paribus generalizations} and thus
covering law models are false \cite{cartwright1980truth}.


\section*{Ceteris Paribus Generalizations}
According to Cartwright the common (and c.l.) view on scientific theories is, that they
express truths about nature and help us explain natural phenomena.
Cartwright argues this is false, since, at least in most cases, there are no covering laws
in most scientific explanation.
This means that they cannot account for most scientific explanations and
\enquote{Covering-law models let in too little.}\cite[2]{cartwright1980truth}.
Thus, she puts more importance to the role of the actual explanation itself,
rather than some supposed law.

But what do scientific theories use then, if not laws of nature?
They use ceteris paribus (c.p.) generalizations, meaning
statements which are applicable only under a specific set of conditions.
This fails to fulfill the condition of universality in the
logical positivist's
definition of natural laws.
As a matter of fact, Cartwright believes there are no
exception less quantitative laws at all.

To understand this argument of Cartwright, let us look at an example
of atomic \enquote{laws} again, take \enquote{Every atomic nucleus is positively charged.}.
It is not-analytic, necessary and general at first glance. But this is only true until we
learn of anti-matter, which has reversed charge, so atoms with negatively charged nuclei.
Hence, the actual law is \enquote{In matter, every atomic nucleus is positively charged}.
The addition of \enquote{In matter} is the ceteris paribus modifier, which makes the statement
true again, but then it is not general anymore and thus no law.

Indeed, even the most successful physical theories, of history and even nowadays,
have some sort of c.p. modifier.
Take Newton's theory of gravity for example,
Almost 200 after its publication and success in explaining the motion of the planets and
the motion of bodies on earth, it was discovered that Mercury was not following
Newton's law precisely. And so a c.p. modifier was added to
Newton's law of gravitation, namely \enquote{Under the condition that the gravity fields are not too strong,
    Newton's laws hold true.}.
Today an even better theory emerged, Einstein's general relativity, which remains
the best theory of gravitation to date. But is this a natural law?
Even though it is still state-of-the-art, it already comes with numerous c.p.
modifiers, for example \enquote{Except at the center of a black hole} or
\enquote{Except for the mass distributions of galaxies} or \enquote{Except at quantum mechanical level}.
Indeed, no accepted physical theory today comes without c.p. modifiers.

\section*{Unknown Laws}
Let's come back to our antimatter-atom example from the last section.
Is the statement about the positivity of an atom's nucleus a law,
if we don't yet know about the existence of antimatter?
Observe that the definition
of a natural law does not include the knowledge of the subject/scientist itself.
A defender of c.l. models would argue, that even if what scientists believed to be a
natural law turns out wrong, there is some better, still unknown covering law, which is a true
natural law.
Cartwright objects this view, with two arguments \cite[5]{cartwright1980truth}.

Firstly, it may be a possibility, that there are no laws.
Maybe nature does not come in handy, easy to understand laws to discover.
This is not a necessity, but there seems to be no reason to discard this
metaphysical possibility.
This questions the necessity of natural laws.
Secondly, the promise of some covering, but still unknown law, is
nothing more than that: a promise. It is of no use to explain
nature, but \enquote{there are at best assurances that explanations are to be had} \cite[5]{cartwright1980truth}.

If it comes in handy, we can choose to believe in (unknown) laws, Cartwright argues, but
we don't do that with any scientific reason.
We cannot test unknown laws and unknown laws can not explain anything.



\section*{Explanations}
What exactly is the goal of science?
As explained, Cartwright does not see it in finding covering laws.
But what if we are in a hypothetical scenario, where humankind has
found all covering laws, with which we can explain everything.
Physicists call this the \enquote{theory of everything}.
Is science over then?
Cartwright argues no, it is still
\enquote{the job of science to tell us what kinds of explanations are admissible}\cite[7]{cartwright1980truth}.
Assume we observe an atom moving away from a positively charged electric source.
How does science explain this?
We have our c.p. law of \enquote{the nucleus of atoms is positively charged},
we know (even analytically, it can be argued), that positive charges
repel each other, and that an atom has negatively charged electrons in its hull,
which can cancel the positive charges in the nucleus.
All these laws are either analytically or come with (numerous) c.p. modifiers, like
the restriction, that we talk about matter, not antimatter.
So the explanation is
\enquote{The atom moved away, because it was positively charged as a whole, since there
    were more positive charges in the nucleus, than negative ones in the hull}.
Cartwright would argue this is the right explanation to give, even if we
can not be absolutely certain, that it is right.
Science does not have to give perfect explanations, and can allow for some oversight,
but we still get some confidence, that this may be the right explanation \cite[7,8]{cartwright1980truth}.
In fact, this explanation can be wrong a lot of the times: Maybe the atom is made of antimatter,
maybe it was not even the positively charged source, which made the atom move.
And even in the case, that this explanation is completely right, we did not even
use known laws, which cover the phenomenon more than our used c.p. laws:
We did not employ the Schrödinger-Equation, nor Quantum-Field-Theory.


\section*{How Does Science Explain?}
I hold two objections with Cartwright's view on
the job of science.
While our explanation for why the atom moves might be
perfectly reasonable for everyday life, it is at best
a first hypothesis in a true scientific explanation.
Although I have chosen a different example than Cartwright herself,
this is also applicable to her dying camellias \cite[6]{cartwright1980truth}.
In practice, science does not only find a reasonable explanation and calls it a day,
a true scientific explanation should explore (all) different
possible explanations to a phenomenon, and rule
as many out as possible, with repeated hypothesis-testing
and experimentation. This is the difference between science and everyday explanation.
Secondly, while I admit that Cartwright gets rid of a lot of metaphysical implications
of natural laws, I do not think we need to go so far, as to get rid of them
completely. After all, scientists use them with great success.
If we just take the "general" in the definition of a natural law to allow
a little more interpretation, we arrive at a powerful set of natural laws,
useful to explain nature.
I argue that a c.p. modifier does not make a law less general. In fact, I argue
it makes the law even more general. A general natural law can include its own
limitations (c.p. modifiers): It is still applicable in general, it will
just tell you in most of the cases, that is of no use.
Cartwright argues, that there is also the possibility that
there might be no covering law, but only if you
define a natural law too precisely.
Even in the case, that some things in the universe turn out to be
completely random, the law \enquote{Things happen with complete randomness.}
is still a general natural law.
I argue c.p. generalizations are natural laws, and they do explain a lot.
Science does not offer any absolute truths, but
in explaining (or describing) nature with c.p. laws at least
some truth can be found, with the restriction the c.p. modifiers provide.
Scientific explanation explores all possible c.p. laws,
and discards laws or adds c.p. modifiers when experiments contradict them.

Take Newton's law of gravitation as a counterexample to Cartwrights argumentation:
After it was discovered, that mercury did not follow Newton's laws, what did
science do?
Scientists predicted a new planet in the solar system, which was just never seen, but
did they rest after this perfectly good explanation?
No, instead Einstein developed general relativity, which could explain the same thing.
After the proposed planet was never seen, this theory was discarded and
we added a c.p. modifier to Newtonian gravity.
But Newtonian gravity still remains a useful, general law of nature to this day,
which in some cases tells you, to not use it.
There is still some truth in Newtonian gravity.
This also has no big metaphysical implications, since science only
claims to approximate absolute truth, if it even exists.
We do not need to believe that natural laws are right, in the sense that
they do not fail. In fact, science is about knowing the limitations of the
very laws it uses.


\section*{Conclusion}
Science uses \emph{ceteris paribus} laws.
Cartwright argues, that these are no laws of nature,
since they are not general with their c.p. modifiers.
While her argument elegantly gets rid of many
metaphysical implications, she removes more
than needed.
C.p. laws are still laws of nature, if we include the
c.p. modifiers, and with that they provide explanations.
These laws of nature provide some portions of the truth,
and are of central importance to scientific
explanation, pragmatically in reality and theoretically.