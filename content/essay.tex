\section*{Introduction}

The role of so-called \emph{Laws of Nature} is of great interest
to philosophy of science.
Intuitively, it seems as if every natural science has laws:
Prominent examples include Newton's \emph{law} of gravitation in physics,
the laws of covalent bonding in chemistry or Mendel's law in biology.
Aside from the fact that these powerful scientific concepts are called \enquote{laws},
many scientists also work under the assumption of laws, or
even actively try to discover new ones.
Thus, it is an important role of the philosophy of science to define and understand
laws of nature and their implication, truth content and usefulness.

Logical positivism defines a law of nature as a universal, necessary statement, which is not analytically true.
This definition is widely agreed upon \cite[57]{philsciencebook}.
The features of universality and being non-analytic are rather intuitive:
After all, the statement \enquote{Some atoms are radioactive.} (which is not universal) doesn't tell us if
Radon is radioactive, whereas \enquote{Every atom contains one or more protons.} (which is analytic)
is already contained in the meaning of the word \enquote{atom}.
(Nomic) Necessity means statements are not accidentally true,
by the concrete circumstances of reality (like \enquote{There are more oxygen .. than gold atoms on earth.}, which is not-necessary).
To distinguish between necessary and not-necessary, we can also require that
the law supports \emph{counterfactuals}, hypothetical if-clauses to see if the law brakes.