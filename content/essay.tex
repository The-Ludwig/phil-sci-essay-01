\section*{Introduction}

The role of so-called \emph{Laws of Nature} is of great interest
to the philosophy of science.
Intuitively, it seems as if every natural science has laws:
Prominent examples include Newton's \emph{law} of gravitation in physics,
the laws of covalent bonding in chemistry or Mendel's law in biology.
Aside from the fact that these powerful scientific concepts are called \enquote{laws},
many scientists also work under the assumption of laws, or
even actively try to discover new ones.
Thus, it is an important role of the philosophy of science to define and understand
laws of nature and their implication, truth content and usefulness.

Logical positivism defines a law of nature as a universal, necessary statement, which is not analytically true.
This definition is widely agreed upon \cite[57]{philsciencebook}.
The features of universality and being non-analytic are rather intuitive:
After all, the statement \enquote{Some atoms are radioactive.} (which is not universal) doesn't tell us if
Radon is radioactive and \enquote{Every atom contains one or more protons.} (which is analytic)
is just a description of the meaning of \enquote{atom}.
(Nomic) Necessity means statements are not accidentally true,
by the concrete circumstances of reality
(like \enquote{The most common atom in the universe is hydrogen.}, which is not-necessary).
To distinguish between necessary and not-necessary, we can also require that
the laws of nature supports \emph{counterfactuals}, which are hypothetical if-scenarios
to test the law
In our not-necessary atom-law example, the counterfactual
\enquote{If all hydrogen atoms were fused to helium, hydrogen would be the most common element in the universe.}
is false, since it is just a coincident law.
Laws of nature are required in \emph{covering law} (c.l.) models, like the \enquote{deductive-nomological (DN) model}
or the \enquote{inductive-statistical (IS) model} \cite[Ch. 3]{philsciencebook}.
This is a class of models which define a scientific explanation as
always using some law of nature, which is relevant to (covers) the to-be-explained phenomenon
(the explanandum).

Nancy Cartwright argues in her 1983 essay \enquote{The Truth Doesn't Explain Much}
that science does not use real laws of nature at all, rather that the explanation
is made using \emph{ceteris paribus generalizations} and thus
covering law models are false \cite{cartwright1980truth}.


\section*{Ceteris Paribus Generalizations}
According to Cartwright the common (and c.l.) view on scientific theories is, that they
express truths about nature and help us explain natural phenomena.
Cartwright argues this is false, since, at least in most cases, there are no covering laws
in most scientific explanation.
This means that they cannot account for most scientific explanations and
\enquote{Covering-law models let in too little.}\cite[2]{cartwright1980truth}.
Thus, she puts more importance to the role of the actual explanation itself,
rather than some supposed law.

But what do scientific theories use then, if not laws of nature?
They use ceteris paribus (c.p.) generalizations, meaning
statements which are applicable only under a specific set of conditions.
This fails to fulfill the condition of universality in the
logical positivist's
definition of natural laws.
As a matter of fact, Cartwright believes there are no
exception less quantitative laws at all.

To understand this argument of Cartwright, let us look at an example
of atomic \enquote{laws} again, take \enquote{Every atomic nucleus is positively charged.}.
It is not-analytic, necessary and general at first glance. But this is only true until we
learn of anti-matter, which has reversed charge, so atoms with negatively charged nuclei.
Hence, the actual law is \enquote{In matter, every atomic nucleus is positively charged}.
The addition of \enquote{In matter} is the ceteris paribus modifier, which makes the statement
true again, but then it is not general anymore and thus no law.
But what if we don't yet know about the existence of antimatter?
Observe that the definition
of a natural law does not include the knowledge of the subject/scientist itself.
A defender of c.l. models would argue, that even if what scientists believed to be a
natural law turns out wrong, there is some better, still unknown covering law, which is a true
natural law.
Cartwright objects this view, saying laws are scarce.

Indeed, even the most successful physical theories, of history and even nowadays,
have some sort of c.p. modifier.
Take Newton's theory of gravity for example,
Almost 200 after its publication and success in explaining the motion of the planets and
the motion of bodies on earth, it was discovered that Mercury was not following
Newton's law precisely. And so a c.p. modifier was added to
Newton's law of gravitation, namely \enquote{Under the condition that the gravity fields are not too strong,
    Newton's laws hold true.}.
Today an even better theory emerged, Einstein's general relativity, which remains
the best theory of gravitation to date. But is this a natural law?
Even though it is still state-of-the-art, it already comes with numerous c.p.
modifiers, for example \enquote{Except at the center of a black hole} or
\enquote{Except for the mass distributions of galaxies} or \enquote{Except at quantum mechanical level}.
Indeed, no accepted physical theories today comes without c.p. modifiers.