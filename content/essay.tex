\section*{Introduction}
This essay starts with a discussion on the role
of \emph{laws of nature} in the philosophy
of science.
I go on to explain \emph{Nancy Cartwright's}
positions on laws of nature expressed in her
1983 essay \enquote{The Truth Doesn't Explain Much}.
Which puts the importance of
laws of nature in science to question,
and thus refuses any \emph{covering law} model
of explanation.
The role of the explanatory process is
key in her scientific model.
I argue that her key observation,
science mostly uses
\emph{ceteris paribus} generalizations to explain, instead of truly general laws, is right,
but that they do fulfill the requirements of laws of nature.
In fact, science uses somewhat general laws of nature all the time,
and in doing so provides some truths about nature.


Intuitively, it seems as if every natural science has \emph{laws of nature}:
A prominent example is Newton's \emph{law} of gravity in physics.
Aside from the fact that many powerful scientific concepts are called \enquote{laws},
many scientists also work under the assumption of laws, or
even actively try to discover new ones.

A logical-positivist might define a \emph{law of nature} in its simplest form as
a universal, necessary statement, which is not analytically true.
This definition\footnote{Many definitions are more elaborate,
    I focused on the most essential parts.}
is widely agreed upon \cite[57]{philsciencebook}.
Laws of nature are required in \emph{covering law} (c.l.) models, like the
deductive-nomological (DN)
or the inductive-statistical (IS) model \cite[Ch. 3]{philsciencebook}.
This class of models requires a scientific explanation to
use at least one law of nature, which is relevant to (covers) the to-be-explained phenomenon.

Nancy Cartwright argues in her 1983 essay \enquote{The Truth Doesn't Explain Much}
that science does not use real laws of nature at all, rather that the explanation
is made using \emph{ceteris paribus generalizations} and thus
covering law models are false \cite{cartwright1980truth}.


\section*{Ceteris Paribus Generalizations}
According to Cartwright the common (and c.l.) view on scientific theories is, that they
express truths about nature and help us explain natural phenomena.
She argues this is false, since there are no covering laws
in most scientific explanations.
This means that they cannot account for most scientific explanations and thus
\enquote{covering-law models let in too little}\cite[2]{cartwright1980truth}.

But what do scientific theories use, if not laws of nature?
They use ceteris paribus (c.p.) generalizations:
statements which are applicable only under a specific set of conditions.
This makes them fail the condition of universality in the
definition of natural laws.
Cartwright believes there are no
exceptionless quantitative laws at all \cite[2]{cartwright1980truth}.

To understand this argument of Cartwright, let us look at an example
of atomic \enquote{laws}, take \enquote{Every atomic nucleus is positively charged.}.
It is not-analytic, necessary and general at first glance. But this is only true until we
learn of anti-matter, which has reversed charge, so atoms with negatively charged nuclei.
Hence, the actual law is \enquote{In matter, every atomic nucleus is positively charged}.
The addition of \enquote{In matter} is the ceteris paribus modifier, which makes the statement
true again, but it does not fulfill universality anymore.

Indeed, even the most successful physical theories,
have some sort of c.p. modifier, e.g.
\enquote{Under the condition that the gravity fields are not too strong,
    Newton's laws hold true.}.
Today an even better theory emerged, Einstein's general relativity, which remains
the best theory of gravitation to date. But is this a natural law?
Physicists are very aware of its limitation,
and apply numerous c.p. modifiers to it, like \enquote{Except at the center of a black hole}
or \enquote{Except at quantum mechanical level}.
Indeed, no accepted physical theory today comes without c.p. modifiers.

\section*{Unknown Laws}
Is the statement about the positivity of an atom's nucleus a law,
if we don't yet know about the existence of antimatter?
A defender of c.l. models would argue, that even if what scientists believed to be a
natural law turns out wrong, there is some better, still unknown covering law, which is a true
natural law.
Cartwright objects this view, with two arguments \cite[5]{cartwright1980truth}.

Firstly, it is a possibility, that there are no laws.
This is not a necessity, but there seems to be no reason to discard this possibility.
Natural laws may not be needed after all.
Secondly, the promises of some still unknown law,
are no explanations,
but \enquote{they are at best assurances that explanations are to be had} \cite[5]{cartwright1980truth}.

If it comes in handy, we can choose to believe in (unknown) laws, Cartwright argues, but
we don't do that with any scientific reason.
We cannot test unknown laws and unknown laws can not explain anything.



\section*{Explanations}
But what if we are in a hypothetical scenario, where humankind has
found enough laws, so we can explain everything
\footnote{Physicists call this the \enquote{theory of everything}.}.
Is science over then?
No, it is still
\enquote{the job of science to tell us what kinds of explanations are admissible}\cite[7]{cartwright1980truth},
Cartwright argues.
Assume we observe an atom moving away from a positively charged electric source.
How does science explain this?
We have our c.p. law of \enquote{the nucleus of matter atoms is positively charged},
we know, that positive charges
repel each other, and that an atom has negatively charged electrons in its hull.
All these laws are either analytically or come with (numerous, implicit) c.p. modifiers.
The explanation is
\enquote{The atom moved away, because it was positively charged as a whole, since there
    were more positive charges in the nucleus, than negative ones in the hull}.
Cartwright would argue this is the right explanation to give, even if we
can't be absolutely certain that it is right.
Science does not have to give perfect explanations, and can allow for some oversight
\cite[7,8]{cartwright1980truth}.
In fact, this explanation can be absolutely wrong: Maybe the atom is made of antimatter,
maybe it was not even the positively charged source, which made the atom move in the first place.
Even in the case, that this explanation is completely right, we never
used the more fundamental laws, which cover the phenomenon:
We did not employ the Schrödinger-Equation, nor Quantum-Field-Theory
and thus weaker c.p. generalizations are used in science to explain.


\section*{How Does Science Explain?}
I hold two objections with Cartwright's view on
the job of science.
While our explanation for why the atom moves might be
perfectly reasonable in everyday life, it is at best
a first hypothesis in a true scientific explanation
\footnote{Although I have chosen a different example than Cartwright herself,
    this is also applicable to her dying camellias \cite[6]{cartwright1980truth}.}.
In reality, science doesn't just find a reasonable explanation and calls it a day,
a true scientific treatise explores (all) different
possible explanations to a phenomenon, and rules
out as many as possible, with repeated hypothesis-testing
and experimentation.
Secondly, while I admit that Cartwright gets rid of a lot of metaphysical implications
of natural laws, I do not think we need to go so far, as to get rid of them
completely. After all, scientists use them with great success.
A c.p. modifier does not make a law less general. In fact, I argue
it makes the law even more general. A natural law can include its own
limitations (c.p. modifiers): It is still applicable in general, it will
just tell you in some, or even most, of the cases that it is of no use.

Cartwright argues, that there is also the possibility that
no covering law might exist at all.
I disagree. Even in the case, that some things in the universe turn out to be
completely random, the law \enquote{Things happen with complete randomness.}
is a natural law.
I argue c.p. generalizations are natural laws, and they do explain a lot.
Science does not offer any absolute truths, but
in explaining (or describing) nature with c.p. laws at least
some truth can be found, with the restrictions the c.p. modifiers provide.

Take Newton's law of gravitation as a counterexample to Cartwright's argumentation:
After it was discovered, that Mercury did not follow Newton's laws, what did
science do?
Scientists predicted a new planet in the solar system, but
did they rest after giving this perfectly good explanation?
No, instead Einstein developed general relativity, which explained the same thing.
After the proposed planet was never seen, the new-planet-theory was discarded, and
we added a c.p. modifier to Newtonian gravity.
But Newtonian gravity still remains a useful, general law of nature to this day,
which in some cases tells you, to not use it.
There is still truth in Newtonian gravity.
This has no big metaphysical implications:
We do not need to believe that natural laws are true,
in the sense that they are always right and applicable.
In fact, science is about knowing the limitations of the
very laws it uses.

\section*{Conclusion}
Science uses \emph{ceteris paribus} laws.
Cartwright argues, that these are no laws of nature,
since they are not general with their c.p. modifiers.
While her argument elegantly gets rid of many
metaphysical implications, she removes more
than needed.
I argue, c.p. laws are still laws of nature, if we include the
c.p. modifiers, and with that they provide explanations.
These laws of nature provide some truth,
and are of central importance to scientific
explanation.